% talk about data

We acquired the academic dataset from Yelp\cite{yelp}. The dataset consists of
\numUserTotal users and \numBusTotal businesses. The businesses are all located
in the vicinity of $30$ different universities in the United States. There are
a total of \numRatingTotal ratings. We initially thought we were going to receive 2 distinct datasets (one for University of Michigan,
 one for Princeton) but received the same one twice.
As an alternative, we partitioned our dataset into two; one set
including only businesses from California (\numBusCA businesses) and one set with all other
businesses (\numBusOthernsp)


%We initially misunderstood the datasets. We
%believed, from the wording on the website, we were obtaining two different
%datasets, one from Michigan and one from Princeton, but we actually received
%the same dataset twice. Our initial plan was to get two different datasets,
%training on the first then finally testing on the second. Unfortunately we
%unknowingly trained on the entire body of data, so we weren't able to test on
%any fresh data. 



%For our application, this isn't as big of a problem as others in practice. This
%is because in practice, we will be able to tune the model on the entire body of
%data fairly often. Because of our optimizations, tests run fairly quickly making
%it possible to adjust the model fairly often. Keeping a set of fresh data to
%test on is important for situations where you wish to have a solution that will
%work on all datasets without any need for tuning and that really isn't our goal.

