\subsection{Collaborative Filtering}

To give predicted ratings we decided to use a form of \emph{Collaborative
Filtering} (CF). CF techniques use the tastes of a large collection of users to
predict the taste of a single user.

A very simple example of collaborative filtering would be to use a method such
as difference squared to measure the distance between the ratings of any two
users. Once we have defined our distance function, if we want to predict a
rating of user $b$ on business $b$ we can do the following: Find the distance
between $u$ and each user with a rating for $b$, choosing the user $v$ with the
shortest distance to $u$. Now we say that since $u$ and $v$ usually have very
similar ratings, $u$ will probably feel the same about $b$ as $v$ does. This is
a very na\"{i}ve method, that can be improved by things like taking more related
users into account and averaging their rating for $b$.

Even with improvements to the previous method, we wouldn't really take the way
people make decisions about businesses into account. It won't be terribly
accurate, and there are a number of situations in which this method won't be
able to get a rating at all. If none of the users related to a user have a
rating for a business we wish to predict for, this simple CF algorithm will be
unable to make any prediction for that business at all. 

\subsection{Singular Value Decomposition}
To correct for these problems we decided to perform a matrix factorization
using singular value decomposition (SVD)\cite{bellkor}. SVD fixes both of the
problems that are mentioned above. It decomposes relationships between users
and businesses into some number of factors that influence the relationship.
This allows SVD to better model why people like particular businesses.
SVD can also use less direct relationships between users, so even if
closely related users don't have a rating for a business, we can still give a
prediction for it.

The goal of SVD is to decompose the given matrix $M$ into
matrices $P$ and $Q$ such that $P \times Q \approx M $. This means that if
matrix $M$ has dimensions $u \times b$, $P$ should be $u \times K$ and $Q$
should be $K \times b$. We know that $b$ and $u$ should be the number of
businesses and users respectively, but what is $K$? In SVD $K$ corresponds to
the number of \emph{latent factors} we believe determine the relationship
between users and businesses.

The only input to the SVD algorithm is a matrix $M$ of size $b \times u$ and the
integer $K$. The output of SVD is then matrices $P$ and $Q$. $P$ contains a
relationship between every user and every latent factor. Similarly $Q$ describes
relationships between businesses and latent factors. These relationships aren't
on the same scale as the input ratings of $M$, but they are relative. A value
of $.9$ between a user and a factor suggests that that user has a stronger
relationship to the factor than a value of $.4$. It is also possible to have a
negative relationship. Because $P \times Q$ is an
approximation for $M$, $P_u \cdot Q_b$ approximates a rating for user $u$ on
business $b$. This is referred to as Standard SVD\cite{bellkor}.

SVD on sparse matrices is calculated using an iterative gradient descent
approach to discover good values for $P$ and $Q$. Using gradient descent allows
us to calculate $P$ and $Q$ using only the observed values and ignoring any
empty values. This means that we initialize $P$ and $Q$ to some values, and
then calculate the distance from $P \times Q$ to $M$. We then move the values
in $P$ and $Q$ in the correct direction to decrease the distance. 
This means that the value of $P \times Q$ should move closer
to $M$ with each iteration. At the end of each iteration we measure the
difference between the previous distance and the current one. When we make a
small enough change, or a predetermined amount of iterations have occurred, the algorithm terminates.

Our goal in each iteration is given some $P$ and $Q$, for every $i$ and $j$ for
which there exists a rating $M_{ij}$ and for every latent variable $k$ from
$0\ldots K$, we should determine a new $P_{ik}$ and $Q_{kj}$. We do this using the
following equations

\[
\begin{array}{c}
P'_{ik}=P_{ik} + \alpha(E_{ij}Q_{kj}-\beta P_{ik}) \\
Q'_{kj}=Q_{kj} + \alpha(E_{ij}P_{ik}-\beta Q_{kj})
\end{array}
\]

\noindent where $E$ is the error between $P \times Q$ and the actual value.
What we are doing here is modifying each user and business association in a
direction that decreases the distance between $M$ and $P \times Q$. The term
multiplied by $\beta$ is the regularization term. This term keeps the values
from changing too quickly relative to how large they already are. This
restricts the absolute values in $P$ and $Q$ from growing too large to avoid
over-fitting the data.

You can see that each iteration is controlled by two additional parameters.
$\alpha$ scales the amount that we increase or decrease the values in $P$ and
$Q$. A larger $\alpha$ can lead to faster convergence, but if $\alpha$ gets too
big, it is possible that we will greatly overshoot the answer and simply
oscillate around it without ever reaching convergence. Similarly, if $\alpha$
is too small, the algorithm will reach convergence much slower than it would
with a more appropriate value for $\alpha$. We also use $\beta$ to control the
effect of our regularization term.  A larger $\beta$ will decrease the chance
of over-fitting but make convergence slower.

\subsubsection{Normalization}

Our ratings data comes from a 5-star system, which everyone can treat slightly
differently. For example, perhaps some users are particularly harsh and rarely
give out 5-star ratings, or others are particularly kind and rarely give any
business less than 3-stars. To account for this we normalize our data by
considering each rating to be comprised of four factors (cf. \cite{bellkor}): 

\begin{description}

  \item[Global Baseline] This is the global average of all ratings on the site
and serves as a baseline for all ratings.

  \item[User Specific] This is the different between the global average and
this users average, and serves to account for the users particular bias.

  \item[Business Specific] Same as \textbf{User Specific}, but for the business

  \item[User-Business Interaction] This is the specific interaction between this
user and this business. This is the term that we try to identify through SVD. 

\end{description}

By attempting to specifically identify the user-business interaction, we remove
the bias in the system, with the users, and with the businesses to more
accurately predict the user's rating of the business. 

\subsubsection{Capping}

People typically reserve the highest and lowest ratings for the perfect and
worst possible business respectively. Currently, a particularly strong factor
could give a very strong positive rating, such as $+7$, and a second factor has
a minor negative influence, such as $-1$, would result in an overall rating of
$+5$. That is, the high positive rating would negate any effects the small
negative rating would have. Thus, we cap the amount of influence each factor
can have on the rating to between $-5$ and $+5$ during the dot-product calculation
in our algorithm. With the cap in place, our
strong $+7$ rating is capped at $+5$ which allows the $-1$ rating to pull our
overall rating away from a perfect score and down to $+4$. \cite{funk}

\subsection{Framework}

Because we actually want to build a usable system on top of our recommendation
engine, we built a prototype website in Django\cite{django}. Django is an MVC
(Model-View-Controller) web framework based on Python. MVC frameworks like
Django provide modularity between the design and functionality of websites
along with convenient abstractions for accessing databases.  Behind the scenes,
a MySQL database is used to hold information about users, businesses, and
ratings.

\subsection{Challenges}

Our initial implementation had two problems: it used a dense matrix
representation, and it was implemented in Python. Because the gradient descent
algorithm only operates on non-zero ratings (we use zero ratings to represent
no rating), a dense matrix representation was extremely expensive, both in
memory and computationally. Because most of the ratings are zeroes, our first
large performance improvement came from switching to a sparse representation.
Instead of creating a 2-dimensional array containing a rating for every
user-business pair, we created a 1-dimensional array where each cell contained a
username, a business name, and a rating. This allowed us to use only a fraction
of the memory, and to save a large number of wasted loop iterations. The dense
matrix representation took up too much memory and completely stalled the program
in the setup stages, so we can't accurately analyze the speedup we got by
switching to a sparse representation.

The next step we took to improve performance was to move our algorithm code into
C++. To do this we used the Boost library \cite{boost} to compile C++ code into a python
plugin. Once this was done we could call our C++ code as if it were python code.
This gave a significant speedup and got us to the point where we could execute
tests in a reasonable amount of time. To improve the speed of tests even more we
implemented task-level parallelism, allowing us to run tests for different
values of $K$ simultaneously.

%In future work, we will look into parallelizing
%the algorithm code. This is an interesting challenge, because the outer loop
%appears to carry dependences. We suspect, however, that the order we update
%cells in $P$ and $Q$ actually won't have any effect on the correctness of the
%algorithm. This is easy to check because the distance should get better at every
%step. As long as that invariant holds, we can be sure our algorithm will
%eventually give a good answer.
%
%The dataset presented some challenges as well. The first challenge we dealt with
%is that some users and businesses had a very limited number of reviews. Users
%and Business with only a single review don't add any information to the
%algorithm, so we removed those. We also added a method for removing users and
%businesses with fewer than an arbitrary number of reviews. The effect of this is
%shown in the results section.
