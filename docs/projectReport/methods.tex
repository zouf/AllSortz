\subsection{Collaborative Filtering}
To give predicted ratings we decided to use a form of \emph{Collaborative
Filtering} (CF). CF techniques use the tastes of a large collection of users to
predict the taste of a single user.

A very simple example of collaborative filtering would be to use a method such
as difference squared to measure the distance between the ratings of any two
users. Once we have defined our distance function, if we want to predict a
rating of user $b$ on business $b$ we can do the following: Find the distance
between $u$ and each user with a rating for $b$, choosing the user $v$ with the
shortest distance to $u$. Now we say that since $u$ and $v$ usually have very
similar ratings, $u$ will probably feel the same about $b$ as $v$ does. This is
a very naive method, that can be improved by things like taking more related
users into account and averaging their rating for $b$.

Even with improvements to the previous method, we don't really take the way
people make decisions about businesses into account. It won't be terribly
accurate, and there are a number of situations in which you won't be able to get
a rating at all. If none of the users related to a user have a rating for a
business we wish to predict for, we will be unable to make any prediction for
that business at all. 

\subsection{Non-negative Matrix Factorization}
To correct for these problems we decided to use non-negative matrix
factorization (NMF). NMF fixes both of the problems that are mentioned above. It
decomposes relationships between users and businesses into some number of
factors that influence the relationship. Because of this it ends up being a much
better model for why people actually like businesses that they like. It can use
less direct relationships between users, so even if closely related users don't
have a rating for a business, we can still give a prediction for it. 

NMF is one of many methods that solve the matrix factorization problem. The goal
of matrix factorization is that given matrix $M$ should be decomposed into
matrices $P$ and $Q$ such that $P \times Q \approx M $. This means that if
matrix $M$ has dimensions $u \times b$, $P$ should be $u \times k$ and $Q$
should be $k \times b$. We know that $b$ and $u$ should be the number and
businesses and users respectively, but what is $k$? In NMF $k$ corresponds to
the number of \emph{latent factors} we believe determine the relationship
between users and businesses.

The only input to the NMF algorithm is a matrix $M$ of size $b \times u$ and the
integer $k$. The output of NMF is then matrices $P$ and $Q$. $P$ contains a
relationship between every user and every latent factor. Similarly $Q$ describes
relationships between businesses and latent factors. These relationships aren't
on the same scale as the input ratings of $M$, if everything worked correctly
they should be smaller. They are relative however. A value of $.9$ between a
user and a factor suggests that that user has a stronger relationship to the
factor than a value of $.4$. Because $P \times Q$ is an approximation for $M$,
$P_u \cdot Q_b$ approximates a rating for user $u$ on business $b$

The equation for calculating NMF unfortunately doesn't have a closed form, so we
need to use an iterative gradient descent approach to discover good values for
$P$ and $Q$. This means that we initialize $P$ and $Q$ to some values, and then
calculate the distance from $P \times Q$ to $M$. We then move the values in $P$
and $Q$ in the correct direction to decrease the distance between $P \times Q$
and $M$. This means that the value of $P \times Q$ should move closer to $M$
with each iteration. At the end of each iteration we measure the difference
between the previous distance and the current one. When we make a small enough
change the algorithm terminates.

%%equations, normalization here

Each iteration is controlled by two additional parameters. $\alpha$ scales the
amount that we increase or decrease the values in $P$ and $Q$. A larger $\alpha$
can lead to faster convergence, but if $\alpha$ gets too big, it is possible that
we will greatly overshoot the answer and simply oscillate around it without ever
reaching convergence, or just reach convergence much slower than we would with a
more appropriate value for $\alpha$. We also use $\beta$ to control the effect
of our normalization term. A larger $\beta$ will decrease the chance of
overfitting but make convergence slower.

\subsection{Framework}

python site, mysql db so on\ldots

\subsection{Challenges}

Our initial implementation had two problems: it used a dense matrix
representation, and it was implemented in Python. Because the gradient descent
algorithm only operates on non-zero ratings, 
