%%This is a very basic article template.
%%There is just one section and two subsections.
\documentclass[12pt]{article}
\usepackage[margin=1in]{geometry}
\usepackage{graphicx}
\usepackage{amssymb}
\usepackage{bbm}
\usepackage{mathtools}
\usepackage{subfig}
\usepackage{fullpage}
\usepackage{tabularx}
\usepackage{amsmath}
\begin{document}
%%To do:
%%___________________
%%  fill in data numbers
%%  sources
%%  explain normalization
%%  more equations in methods (only if we need to fill space)
%%  numbers for paring
%%  numbers for normalization

\section{Problem}
``What should I do when I'm bored?''  is a question that almost everyone has
asked at one time. A brief look at common Google searches highlights the demand
for activity ideas. We started this project to be able to answer this question
for our users. The only information that should be required from a user is 
a number of ratings of other businesses. Using only that information we want to
be able to give a predicted rating for any other business. We should also be
able to give a user a list of their highest predicted ratings as a good answer
to ``what should I do?''. Additionally we would like to give a user some insight
into why we are recommending a business. This will help to give confidence in
our recommendations, and will also give the user some expectation of what to
expect when they arrive at the business.

A running instance of our solution can be found at http://zoufbox.
Note, you must access this from within the CS department.

\newcommand{\bestRMSE}{0.9307 }
\newcommand{\bestRMSEnsp}{0.9307}
\newcommand{\bestK}{48 }
\newcommand{\bestNetflixRMSE}{0.8712 }
\newcommand{\bestNetflixRMSEnsp}{0.8712}
\newcommand{\netDiff}{0.06 }

\newcommand{\numBusCA}{707 } 
\newcommand{\numBusTotal}{2,452 }
\newcommand{\numUserTotal}{??? }

\newcommand{\numRatingCA}{8,970 } 
\newcommand{\numRatingTotal}{28,310 }

\section{Data}
We acquired the academic dataset from Yelp. The rating consists of \numUserTotal
users and \numBusTotal businesses. The businesses are all located in the vicinity of $30$ different
universities in the United States. There are a total of \numRatingTotal ratings. We initially
misunderstood the datasets. We thought we were obtaining two different
datasets, one from Michigan and one from Princeton, but we actually received the
entire dataset twice. Our initial plan was to get two different datasets,
training on the first then finally testing on the second. Unfortunately we
unknowingly trained on the entire body of data, so we weren't able to test on
any fresh data. 

For our application, this isn't as big of a problem as others in practice. This
is because in practice, we will be able to tune the model on the entire body of
data fairly often. Because of our optimizations, tests run fairly quickly making
it possible to adjust the model fairly often. Keeping a set of fresh data to
test on is important for situations where you wish to have a solution that will
work on all datasets without any need for tuning and that really isn't our goal.



\section{Methods}
\subsection{Collaborative Filtering}
To give predicted ratings we decided to use a form of \emph{Collaborative
Filtering} (CF). CF techniques use the tastes of a large collection of users to
predict the taste of a single user.

A very simple example of collaborative filtering would be to use a method such
as difference squared to measure the distance between the ratings of any two
users. Once we have defined our distance function, if we want to predict a
rating of user $b$ on business $b$ we can do the following: Find the distance
between $u$ and each user with a rating for $b$, choosing the user $v$ with the
shortest distance to $u$. Now we say that since $u$ and $v$ usually have very
similar ratings, $u$ will probably feel the same about $b$ as $v$ does. This is
a very naive method, that can be improved by things like taking more related
users into account and averaging their rating for $b$.

Even with improvements to the previous method, we don't really take the way
people make decisions about businesses into account. It won't be terribly
accurate, and there are a number of situations in which you won't be able to get
a rating at all. If none of the users related to a user have a rating for a
business we wish to predict for, we will be unable to make any prediction for
that business at all. 

\subsection{Non-negative Matrix Factorization}
To correct for these problems we decided to use non-negative matrix
factorization (NMF). NMF fixes both of the problems that are mentioned above. It
decomposes relationships between users and businesses into some number of
factors that influence the relationship. Because of this it ends up being a much
better model for why people actually like businesses that they like. It can use
less direct relationships between users, so even if closely related users don't
have a rating for a business, we can still give a prediction for it. 

NMF is one of many methods that solve the matrix factorization problem. The goal
of matrix factorization is that given matrix $M$ should be decomposed into
matrices $P$ and $Q$ such that $P \times Q \approx M $. This means that if
matrix $M$ has dimensions $u \times b$, $P$ should be $u \times k$ and $Q$
should be $k \times b$. We know that $b$ and $u$ should be the number and
businesses and users respectively, but what is $k$? In NMF $k$ corresponds to
the number of \emph{latent factors} we believe determine the relationship
between users and businesses.

The only input to the NMF algorithm is a matrix $M$ of size $b \times u$ and the
integer $k$. The output of NMF is then matrices $P$ and $Q$. $P$ contains a
relationship between every user and every latent factor. Similarly $Q$ describes
relationships between businesses and latent factors. These relationships aren't
on the same scale as the input ratings of $M$, if everything worked correctly
they should be smaller. They are relative however. A value of $.9$ between a
user and a factor suggests that that user has a stronger relationship to the
factor than a value of $.4$. Because $P \times Q$ is an approximation for $M$,
$P_u \cdot Q_b$ approximates a rating for user $u$ on business $b$

The equation for calculating NMF unfortunately doesn't have a closed form, so we
need to use an iterative gradient descent approach to discover good values for
$P$ and $Q$. This means that we initialize $P$ and $Q$ to some values, and then
calculate the distance from $P \times Q$ to $M$. We then move the values in $P$
and $Q$ in the correct direction to decrease the distance between $P \times Q$
and $M$. This means that the value of $P \times Q$ should move closer to $M$
with each iteration. At the end of each iteration we measure the difference
between the previous distance and the current one. When we make a small enough
change the algorithm terminates.

%%equations, normalization here

Each iteration is controlled by two additional parameters. $\alpha$ scales the
amount that we increase or decrease the values in $P$ and $Q$. A larger $\alpha$
can lead to faster convergence, but if $\alpha$ gets too big, it is possible that
we will greatly overshoot the answer and simply oscillate around it without ever
reaching convergence, or just reach convergence much slower than we would with a
more appropriate value for $\alpha$. We also use $\beta$ to control the effect
of our normalization term. A larger $\beta$ will decrease the chance of
overfitting but make convergence slower.

\subsection{Framework}

python site, mysql db so on\ldots

\subsection{Challenges}

Our initial implementation had two problems: it used a dense matrix
representation, and it was implemented in Python. Because the gradient descent
algorithm only operates on non-zero ratings, 

\section{Evaluation}
We evaluated our technique in phases. First, we generated fake data that validates the correctness of the model, then we evaluated a small portion of the 
dataset, and finally evaluated on the full dataset.

\subsection{Generated Data}
The core algorithm relies on finding a $P$ and $Q$ matrix that relate users and businesses to latent factors (the columns of $P$ and $Q$). 
Figuring out specifically how a user (u) feels about a business(b) is just a matter of taking the dot product of $P_u$ and $Q_b$. 
Therefore, we started evaluation by randomly generating a $P$ and $Q$ matrix, and then creating the rating table based on the 
matrices. Running our algorithm on this data yielded similar $P$ and $Q$ matrices, which indicated that the algorithm was (fundamentally) doing what
we expected it to do.

\subsection{Real Data}
However, using generated data shows nothing about the validity of our model. In order to show just how effective our model would be, we needed to evaluate on the actual
businesses data from Yelp. We started evaluation by taking dense chunks of data from the Yelp set. Thus, we'd only have a couple of thousand ratings (instead of hundreds of thousands). 
This step was critical to expose inefficiencies in our calculations (indeed, once we started doing this, we immediately realized the need to switch our code from Python to a C++ module).

For training, we removed California's businesses ratings from the Yelp data set. This meant removing roughly \numBusCA businesses from \numBusTotal total businesses and \numRatingCA from \numRatingTotal ratings. Figure~\ref{fig:nocal} shows a plot of RMSE vs. K for all data excluding California's. For the evaluation step, we ran our algorithm on ratings for California businesses only. Figure~\ref{fig:cal} for a plot of RMSE vs. K for ratings only on California businesses. No parameters were changed between the training and evaluation run. The graphs show that the K-values have roughly the same effect on RMSE in both cases. Thus, a K-value that gives a poor RMSE value in the data excluding California gives a good RMSE value in the data for California (and vice versa). This result indicates that our model works well on data which we did not train or tune with.

\begin{figure}[ht!]
	\centering
%%	\includegraphics[scale=0.5]{figures/nocal.pdf}
	\caption[]{A plot of RMSE vs. K for all ratings excluding California businesses}
	\label{fig:nocal}
\end{figure}


\begin{figure}[ht!]
	\centering
	%%\includegraphics[scale=0.5]{figures/cal.pdf}
	\caption[]{A plot of RMSE vs. K for all ratings on California businesses}
	\label{fig:cal}
\end{figure}

\begin{figure}[ht!]
	\centering
	%%\includegraphics[scale=0.5]{figures/norm.pdf}
	\caption[]{A plot of RMSE vs. K for all ratings excluding California businesses without normalization}
	\label{fig:norm}
\end{figure}

We also evaluated the effect of normalization by running the same tests as above on businesses excluding California with and without a normalization factor. Figure~\ref{fig:norm} shows a plot of the same ratings and businesses as Figure~\ref{fig:nocal} without normalization.

\subsection{Interpreting Results}
As we increase the K-values, the RMSE value approaches a minimum value which it rarely dips below. For us, we were able to achieve an RMSE of \bestRMSE when K=\bestK. Increasing K beyond this did not improve RMSE, and would even cause over-fitting. What's particularly interesting about our result is that it shows predictions across a wide variety of businesses-genres can be useful in predicting interests in unrelated genres. For example, predictions for clothing shops can be used to help predict where you might like to go out to eat. We initially thought that using unrelated businesses to predict one another would yield to inaccurate results. However, when we excluded non-food related ratings from the dataset, we achieved worse RMSE results. See Figure~\ref{fig:foodnoly} for a plot.

We also compared our results to the winners of the Netflix prize, \cite{netprize}. Unlike our technique, the winners of the Netflix prize used a combination of different predictors working in unison to build a prediction engine. Ideally, we would build additional prediction engines to supplement ours. However, we were not able to do this due to time constraints. Overall, the winners of the Netflix prize achieved an RMSE value of \bestNetflixRMSE, which is roughly ~0.05 from our best RMSE of \bestRMSE. The difference between the accuracy of our results and the Netflix results is likely due to the fact that movies are much more controlled than businesses. Businesses are effected by location, price, quality of service, and hours of operation. Movies are always available and are displayed in a roughly uniform way. 


\section{Error Analysis}
Error analysis to go in here


\begin{thebibliography}{9}

\bibitem{netprize}
  Edwin Chen,
  \emph{Winning the Netflix Prize: A Summary}.
  http://blog.echen.me/2011/10/24/winning-the-netflix-prize-a-summary/
\end{thebibliography}
\end{document}
